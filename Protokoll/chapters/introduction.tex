
\chapter{Introduction (Manisch)}
\label{sec:Introduction}
If we make comparison is the technology what is going in now a days and what was going 50 years back ,we can able to see radical shift from mechanic to mechatronic systems. Mechatronic systems are usually made up of functional modules, generally consisting of mechanical basic systems (the process), actuators, sensors, signal evaluation processors (controllers) and a user interface device.
Mechanical and electronic components are become smaller then we started use Microsystems technologies. The development of microelectromechanical systems (MEMS), requires very specific expertise and knowledge in areas mechatronics and micro technology. 
To improve knowledge and expertise in MEMS we were given the task to develop a suitable pressure sensor measurement system and choose principle for measuring the occurring level of pressure.Designing a sensor is a many step process. It comprises of many parts from modelling, simulation, fabrication and finally testing it. The individual steps in the system development include,
\begin{itemize}
	
	\item Preparation of a preliminary system design
	\item Creation of a 3D CAD model
	\item Analysis of the mechanical properties in a FEM program
	\item Simulation of the manufacturing process
	\item Characteristics of the sensor
	\item Simulation and design of the electronic circuit
	\item  Structuring the overall system
	\item Testing the sensors and the electronic system
\end{itemize}

To achive all the task several tools are required.Than finally we will get our result.

\section{Zielsetzung und Gliederung}
\label{sec:Zielsetzung}
The first task was to determine the preliminary characteristics of the sensor. The need was to design a sensor for monitoring the gas pressure in a low-pressure pneumatic piping system. There were also other details mentioned. Such as, the maximum pressure in the system is 1.2 bar. So, a sensor with a range of 0-1 bar with a measurement accuracy of -50 mbar to +50 mbar is needed. The pressure levels are quasi-static, so the dynamic range of the measuring system can be neglected. The pressure sensor cannot exceed dimensions of 10x10 mm, as it has to be integrated into an existing housing. An output voltage of 0-1V is needed to represent the equivalent applied pressure in bar. The power supply provides a supply voltage of 12V. The marketing department is expecting the annual demand of 2,000,000 units. We have to consider the number of units and side by side reduce the production cost per unit. 


\section{Geschichte/Literatur}
\label{sec:Geschichte_Literatur}
A pressure sensor is a device for pressure measurement of gases or liquids. Pressure is an expression of the force required to stop a fluid from expanding, and is usually stated in terms of force per unit area. A pressure sensor usually acts as a transducer; it generates a signal as a function of the pressure imposed. For the purposes of this article, such a signal is electrical. Pressure sensors are used for control and monitoring in thousands of everyday applications. Pressure sensors can also be used to indirectly measure other variables such as fluid/gas flow, speed, water level, and altitude.Pressure measurement and control is the most used process variable in the process control industry many segments. In addition, through the pressure it is possible infer a series of other process variables, like level, volume, flow and density. This article will cover the main technologies of the most important technologies used in pressure sensors, as well as some details concerning pressure transmitter installations, market and trends.
Pressure Measurement with little bit about the history, for many years pressure measurement has attracted the interest of science. At the end of the XVI century, the Italian Galileo Galilei (1564-1642) was granted the patent for a water pump system used on irrigation. (As a curiosity: in 1592, using simply a test tube and a water basin, Galileo assembled the first thermometer. The tube was turned upside down and half immersed in the water, so, when the air inside the tube cooled, the volume decreased and the water raised. When the air warmed, the volume raised and the water was forced out. Therefore, the water level measured the air temperature.) The core of its pump was a suction system that could raise the water at a maximum of 10 meters. He never knew the reason for this limit, which motivated other scientists to study this phenomenon. 
In 1643, the Italian physicist Evangelista Torricelli (1608-1647) invented the barometer, with which he could evaluate the atmospheric pressure, i.e., the force of the air over the earth surface. He performed an experiment by filling a 1 meter tube with mercury, sealed at one end and immersed in a tub with mercury at the other. The mercury column invariably immersed in the tube approximately 760 mm. Without knowing exactly the reason for this phenomenon, he attributed it to a force produced at the earth surface. Torricelli also concluded that the space left by the mercury at the beginning of the tube was empty and named it ?vacuum?. Five years later, French physicist Blaise Pascal used the barometer to show that the air pressure was smaller at the top of the mountains. 
In 1650, German physicist Otto Von Guericke developed the first efficient air pump, with which Robert Boyle carried out compression and decompression tests and 200 years later French physicist and chemist Joseph Louis Gay-Lussac, determined that the pressure of a gas confined at a constant volume is proportional to its temperature. 
In 1849, Eug�ne Bourdon was granted the Bourdon Tube patent, used until today in relative pressure measurements. In 1893, E.H. Amagat used the dead-weight piston on pressure measurements.The entire pressure measurement system is constituted by a primary element, which will be in direct or indirect contact with the process where the pressure changes occur, and a secondary element (the pressure transmitter) whose task will be translating the change in measurable values for use in indication, monitoring and control.
Generally, sensors are classified according to the technique used on the mechanical pressure over a proportional electronic signal. Here our main point of discussion is Piezo-resistivity. It refers to the alteration of the electric resistance with the deformation/contraction as a result of the applied pressure. They are mostly formed by crystal elements (strain gauge) interconnected in bridge (wheatstone) with other resistors that provide zero adjustment, sensitivity and temperature compensation. The construction material varies according to the manufacturer and today solid state sensors are easy to find. Disadvantages: limited operation temperature range, applicable on low pressure ranges because they generate a very low, unstable, excitation signal. Currently there is the so-called film transducer, which is made from steam deposition or the injection of strain-gauge elements directly on a diaphragm, minimizing the instability due to the use of adhesives on the alloy of bonded wire models. The great edge is that it produces a higher level electric signal, however totally vulnerable at high temperatures, because the temperature affects the adhesive material used when sticking the silicon to the diaphragm. Several techniques based on the production of piezoresistive silicon sensors (silicon substrate) are emerging, however susceptible to signal degradation in function of the temperature and require compensation from complicated circuits, error minimization and zero sensitivity. They are totally unviable in applications subject to long high temperature periods, as the diffusion degrades the substrates in these conditions.