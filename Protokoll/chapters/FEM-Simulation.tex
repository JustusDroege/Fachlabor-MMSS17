\chapter{FEM-Simulation}
\label{sec:FEM-Simulation}

\begin{figure}[hb]
	\centering
	\begin{tikzpicture}\centering
	\begin{axis}[,xlabel={x / $\mu$m},ylabel={$\sigma$ / Pa},legend pos=north east,legend columns=1,xtick distance=500]
	\pgfplotstableread{chapters/Daten/ohneBossStat.dat}\data
	
	\addplot [color1] table [x=X,y=sigma] {\data};
	
	\end{axis}
	\end{tikzpicture}
	\caption{Spannungsverlauf einer Membran ohne Boss bei \unit[1]{bar}/strain curve of a membrane without boss at \unit[1]{bar}}\label{pl:ohneBoss}
\end{figure}

\begin{figure}[hb]
	\centering
	\begin{tikzpicture}\centering
	\begin{axis}[,xlabel={x / $\mu$m},ylabel={$\sigma$ / Pa},legend pos=north east,legend columns=1,xtick distance=500]
	\pgfplotstableread{chapters/Daten/mitBossStat.dat}\data
	
	\addplot [color1] table [x=X,y=sigma] {\data};
	
	\end{axis}
	\end{tikzpicture}
	\caption{Spannungsverlauf einer Membran mit Boss bei \unit[1]{bar}/ strain curve of a membrane with boss at \unit[1]{bar}}\label{pl:mitBoss}
\end{figure}


\section{Einfluss der Membrandicke auf die maximale Spannung (Viviane Bremer)}
\label{sec:Membrandicke}

In diesem Abschnitt wird der Einfluss der Membrandicke auf die auftretende mechanische Spannung in der Membran betrachtet. Hierf�r werden folgende Dicken genutzt: \unit[15]{$\mu$m}, \unit[25]{$\mu$m}, ,\unit[35]{$\mu$m}.

%\begin{itemize}
%	\item[a.] \unit[15]{$\mu m$}
%	\item[b.] \unit[25]{$\mu m$}
%	\item[c.] \unit[35]{$\mu m$}	
%\end{itemize}

Zur Analyse wird der anliegende Druck f�r alle drei Membrandicken in \unit[0.2]{bar}-Schritten von \unit[0.2]{bar} bis \unit[1]{bar} erh�ht und die resultierende mechanische Spannung ermittelt. Die Verl�ufe sind in Abbildung \ref{pl:Membrandicken} dargestellt. Die Bruchspannung $\sigma_{Br}$ von Silizium betr�gt \unit[830]{MPa} und sollte w�hrend des Sensorbetriebs nicht �berschritten werden. Zur besseren Visualisierung ist sie als Konstante im Graphen dargestellt.

\begin{figure}[hb]
	\centering
	\begin{tikzpicture}\centering
	\begin{axis}[width=0.8\textwidth, xlabel={p / bar},ylabel={$\sigma$ / Pa},legend pos=north west,legend columns=1,xtick distance=0.2]
	\pgfplotstableread{chapters/Daten/Membran.dat}\data

	\addplot [red] table [x=p415,y=bruch415] {\data};
	\addlegendentry{$\sigma_{Br}$}
	
	\addplot [color1] table [x=p415, y=sigx415] {\data};
	\addlegendentry{$\sigma_x$: \unit[35]{$\mu$m}}
	
	\addplot [color2] table [x=p415,y=sigz415] {\data};
	\addlegendentry{$\sigma_z$: \unit[35]{$\mu$m}}
	
	\addplot [color4] table [x=p415,y=sigx425] {\data};
	\addlegendentry{$\sigma_x$: \unit[25]{$\mu$m}}

	\addplot [color3] table [x=p415,y=sigz425] {\data};
	\addlegendentry{$\sigma_z$: \unit[25]{$\mu$m}}
	
	\addplot [blue] table [x=p415,y=sigx435] {\data};
	\addlegendentry{$\sigma_x$: \unit[15]{$\mu$m}}
	
	\addplot [yellow] table [x=p415,y=sigz435] {\data};
	\addlegendentry{$\sigma_z$: \unit[15]{$\mu$m}}
	
	\end{axis}
	\end{tikzpicture}
	\caption{Spannungsverlauf der drei Membrandicken}\label{pl:Membrandicken}
\end{figure}

Membran a erreicht die Bruchspannung schon bei einem Druck von \unit[0.439]{bar}, welcher im ben�tigten Messbereich liegt. Im Falle von Membran b wird sie knapp oberhalb des Messbereichs bei \unit[1.05]{bar} erreicht. Die Bruchspannung bei Membran c entspricht einem anliegenden Druck von \unit[2]{bar}, welcher deutlich oberhalb des Messbereiches ist. Hieraus lassen sich Erkenntnisse f�r das Sensorverhalten gewinnen. Membran a ist zu d�nn f�r diese Messaufgabe, da sie innerhalb des Messbereichs rei�t. Jedoch ist Membran c auch nicht akzeptabel, da durch die gr��ere Membrandicke niedrigere Spannungen auftreten im Vergleich zu Membran b. Diese bildet einen guten Kompromiss f�r die gew�nschte Aufgabe.


\section{Einfluss der Membranabmessungen auf den Spannungsverlauf}
\label{sec:Membranabmessungen}

\subsection{Einfluss der Bossgr��e (Viviane Bremer)}
\label{subsec:Bosssize}

Zur Analyse des Einflusses der Bossgr��e wird die Membrangr��e auf \unit[3600]{$\mu$m} und die Membrandicke auf \unit[25]{$\mu$m} gesetzt. Die Bossgr��e wird folgenderma�en variiert: \unit[800]{$\mu$m}, \unit[1000]{$\mu$m}, \unit[1200]{$\mu$m}.

Die Spannungsverl�ufe und Verschiebungen sind in Abbildung \ref{pl:Bosssize} dargestellt. Bei einer Bossgr��e von \unit[800]{$\mu$m} treten Spannungen von bis zu \unit[274]{MPa} auf. Eine Vergr��erung des Bosses auf \unit[1000]{$\mu$m} f�hrt zu einer maximalen Spannung von  \unit[178]{MPa}. Der gr��te Boss senkt die auftretende Spannung auf \unit[128]{MPa}. Des Weiteren ist zu sehen, dass eine Vergr��erung des Bosses zu einer Angleichung der Spannungsspitzen f�hrt und die Dehnung der Membran von \unit[374]{$\mu$m} auf \unit[77]{$\mu$m} senkt.

\begin{figure}[hb]
	\centering
	\begin{tikzpicture}\centering
	\begin{axis}[,xlabel={x / $\mu$m},ylabel={$\sigma$ / Pa},legend pos=north east,legend columns=1,xtick distance=500]
	\pgfplotstableread{chapters/Daten/MechSpannung.dat}\data
	
	\addplot [color1] table [x=X1,y=sigma1] {\data};
	\addlegendentry{$\sigma$: \unit[800]{$\mu$m}}
	
	\addplot [color2] table [x=X2,y=sigma2] {\data};
	\addlegendentry{$\sigma$: \unit[1000]{$\mu$m}}
	
	\addplot [color3] table [x=X3,y=sigma3] {\data};
	\addlegendentry{$\sigma$: \unit[1200]{$\mu$m}}
	
	\end{axis}
	\end{tikzpicture}
	\caption{Spannungsverlauf der drei Bossgr��en}\label{pl:Bosssize}
\end{figure}