\chapter{Zusammenfassung und Ausblick (Manish)}
\label{sec:Zusammenfassung}
In this laboratory the pressure sensor with corresponding to a evaluation circuit was characterized. The sensor should be:

\begin{itemize}
	\item connected to the evaluation circuit in a pressure range from 0 bar to 1 bar
	\item one linear Output a proportional voltage signal from 0 to 1 V
	\item 1 V is the reference voltage for the circuit and a $\pm$ 12 V balanced voltage
\end{itemize}

Using an FEM simulation, sensors with different pressure membrane structures were used. The tension in the membrane is examined. A sensor with a Boss membrane structure and a membrane thickness of 35 micrometer was a compromise between a sufficient Pressure range and high mechanical stress.

In the second step, the sensor structure of a prefabricated sensor was examined with a microscope. Subsequently, the sensitivity of the sensor was measured with a measuring setup and compared with other sensor structures. A transversal design was the most sensitive in comparison.

In a simulation with the program PSpice, a circuit could be designed that supports the Measured offsets of the sensor to be measured, and the signal is sufficient strengthened. The available capacitors were used in the circuit, Potentiometers and the resistors of the E24 series.

From the initial tests, the error data were collected and the necessary compensation calculations were made. A circuit was created using PSpice software. Then a circuit schematic for the PCB was created using EAGLE software with the error corrections and a PCB was created. The given sensor was mounted on the circuit to obtain the best results. The specifications given at the start were met with the new PCB and the resultant signal was found good.

A board layout was created for this circuit with the help of Eagle software. The board dimensions were 50 mm x 80 mm. The created board layout was created using a Exposure process transfers the structure to a photoactive circuit board. Through a Subsequent etching process, the copper conductor tracks are etched out on the circuit board. By the Soldering of the components previously determined in PSpice simulation was the evaluation circuit manufactured.

The previously examined sensor was measured again with this circuit. By setting of the potentiometers, the offset values and the gains could be exactly adjusted. Thus, the electronic evaluation system with the sensor fulfils the predefined Specifications. In further developments, the linearity of the sensor could be improved. This characteristic is not corrected by the evaluation circuit so that the signal generated by the sensor can only be measured in the Offset and gain.