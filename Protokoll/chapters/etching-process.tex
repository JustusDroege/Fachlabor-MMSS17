\chapter{Etching Process}
\label{sec:EtchingProcess}


\section{Der Reaktionsprozess beim Silizium�tzen (Viviane Bremer)}
F�r die Erstellung der Sensormembran muss zun�chst der �tzprozess mit KOH charakterisiert werden. Die Reaktion besteht aus der Oxidation und einer Reduktion des Siliziums. Somit ergibt sich die Reaktionsgleichung zu

\begin{equation}
	Si + 2OH^- + 2H_2O \longrightarrow SiO_2(OH)_2^{2-} + 2H_2. \label{eq:reaction}
\end{equation}

Zur Berechnung des zu �tzenden Volumens der Membran wird folgende Volumenformel eines Pyramidenstumpfes, 

\begin{equation}
	V = \frac{h}{3}(A_1 + \sqrt{A_1 A_2} + A_2), \label{eq:Pyramide}
\end{equation}
 

ben�tigt. F�r die Fertigung wird ein 4''-Wafer mit einer Dicke von \unit[450]{$\mu$m} genutzt. Die zu erzeugende Membran ohne Boss hat eine Breite von \unit[4000]{$\mu$m} und eine Dicke von \unit[25]{$\mu$m}. Dies f�hrt zu einer �tzh�he von

\begin{equation}
	h = 450\,\mu m - 25\,\mu m = 425\,\mu m.
\end{equation}

Die Fl�che $A_1$ bestimmt sich mit der Membranbreite zu

\begin{equation}
	A_1 = a_1^2 = 4000\,\mu m \cdot 4000\,\mu m = 16\,mm^2.
\end{equation}

Fl�che $A_2$ ergibt sich mit

\begin{align}
	\tan(54,7�) &= \frac{425\,\mu m}{\Delta a} \\
	\Delta a &= \frac{425\,\mu m}{\tan(54,7�)} = 300\,\mu m\\
	a_2 &= a_1 - 2 \Delta a = 3400\,\mu m
\end{align}

zu

\begin{equation}
	A_2 = a_2^2 = 11,56\,mm^2.
\end{equation}

Dies eingesetzt in Gleichung \ref{eq:Pyramide} ergibt f�r das zu �tzende Volumen

\begin{equation}
	V_{Si} = \frac{0.425\,mm}{3}(16 + \sqrt{16 \cdot 11.56} +11.56)\,mm^2 = 5.831\,mm^3.
\end{equation}

Die Dichte $\rho_{Si}$ von Silizium betr�gt \unitfrac[0.002336]{g}{$mm^3$}. Dies f�hrt mit dem errechneten Volumen zu einer Masse von

\begin{equation}
	m_{Si} = \rho V_{Si} = 0.0137\,g
\end{equation}

Mit der Molmasse von Silizium, $M_{Si} = $\unitfrac[28.09]{g}{mol}, ergibt sich die Stoffmenge zu

\begin{equation}
	n_{Si} = \frac{m_{Si}}{M_{Si}} = 4.877 \cdot 10^{-4} mol.
\end{equation}

Wie in Gleichung \ref{eq:reaction} zu sehen, entstehen bei der Reaktion zwei Teile Wasserstoff. Somit ergibt sich f�r die Stoffmenge des Wasserstoffs

\begin{equation}
	n_{H_2} = 2n_{Si} = 2 \cdot 4.877 \cdot 10^{-4}\,mol = 9.754 \cdot 10^{-4}\,mol.
\end{equation}

Damit kann das Volumen mit dem molaren Normvolumen des idealen Gases $V_{m,0}=\unitfrac[22.414]{l}{mol}$ zu

\begin{equation}
	V_{H_2} = V_{m,0} \cdot n_{H_2} = 22.414\,\unitfrac{l}{mol} \cdot 9.754 \cdot 10^{-4}\,mol = 0.0218\,l
\end{equation}

bestimmt werden. Da dies ein sehr geringes Volumen ist, reicht eine gute Bel�ftung als Vorsichtsma�nahme aus. Eine weitere Ma�nahme w�re noch die Beseitigung von Z�ndquellen.

Bei dem �tzprozess werden des Weiteren zwei Teile Wasser umgesetzt. Das Volumen entspricht somit

\begin{align}
	n_{H_2O} &= n_{H_2} = 9.754 \cdot 10^{-4}\,mol \\
	m_{H_2O} &= M_{H_2O} \cdot n_{H_2O} = \unitfrac[18]{g}{mol} \cdot 9.754 \cdot 10^{-4}\,mol = 0.0175\,g\\
	V_{H_2O} &= \dfrac{m_{H_2O}}{\rho_{H_2O}} = \dfrac{0.0175\,g}{1 \unitfrac{g}{cm^3}} = 0.0175\,cm^3 = 0.0175\,ml.
\end{align}

Der �tzprozess soll mit einer 40\%-ige KOH-L�sung auf Basis von zwei Litern Wasser durchgef�hrt werden. Das bedeutet, dass

\begin{equation}
	m_{KOH} = 0.4 m_{H_2O} = 0.4 \cdot \unit[2]{kg} = \unit[800]{g}
\end{equation}

KOH-Salz ben�tigt wird. Dies entspricht einer Stoffmenge von

\begin{equation}
	n_{KOH} = \dfrac{m_{KOH}}{M_{KOH}} = \dfrac{\unit[800]{g}}{\unitfrac[56.11]{g}{mol}} = \unit[14.258]{mol}.
\end{equation}