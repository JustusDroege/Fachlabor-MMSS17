\chapter{Preliminary Design}
\label{sec:Preliminary Design}



\section{Wert und Einheit}
\label{sec:Unit}
Viele Einheiten lassen sich sch�ner darstellen mit dem "`Tag"' \verb|\unit[]{}| beziehungsweise \verb|\unitfrac[]{}{}|. Siehe den Vergleich: ohne 1~m oder mit \unit[1]{m} bzw. ohne 1~m/sec oder mit \unitfrac[1]{m}{sec}.

\section{�berschrift}
\label{sec:Quelltext}
Text Text Text Text Text Text Text Text Text Text Text Text Text Text Text Text Text Text Text Text Text Text Text Text Text Text Text Text Text Text Text Text Text Text Text Text Text Text Text Text Text Text Text Text Text Text Text Text Text Text Text Text Text Text Text Text Text Text Text Text Text Text Text Text Text Text Text Text Text Text Text Text

\section{Anforderungen an einen Drucksensor (Viviane Bremer)}
\label{sec:Anforderungen}

%F�r die Auswertung des morphologischen Kastens muss zun�chst eine Anforderungsliste erstellt werden. 
Im Rahmen der Entwicklung ist es wichtig, Anforderungen an das Produkt aufzustellen. Dies geschieht mit Hilfe einer Anforderungsliste in der au�erdem in Fest-, Mindest- und Wunschanforderungen unterschieden wird. Mit Hilfe der Anforderungsliste kann schlie�lich der morphologische Kasten ausgewertet werden, um eine m�gliche L�sung zu finden. In diesem Fall soll, wie zuvor schon beschrieben, ein Drucksensor zur �berwachung des Gasdrucks in einer Niederdruck-Pneumatikleitung entwickelt werden. Dieser Druck liegt gew�hnlich zwischen \unit[0]{bar} und \unit[1]{bar}, kann jedoch auf maximal \unit[1,2]{bar} ansteigen. Die Messgenauigkeit sollte hierbei mindestens bei \unit[$\pm 50$]{mbar} liegen. Eine �nderung des Druckes verl�uft im vorliegenden Anwendungsfall sehr langsam und kann somit als quasi-statisch angesehen werden. Da der Sensor in eine bestehendes Geh�use integriert wird, d�rfen seine Abma�e \unit[10x10]{mm} nicht �berschreiten. Das Ausgangssignal soll einer Ausgangsspannung von \unit[0]{} bis \unit[1]{V} entsprechen und somit den anliegenden Druck in \unit[]{bar} repr�sentieren. Zur Spannungsversorgung steht eine symmetrische Spannung von \unit[12]{V} und eine Referenzspannung von \unit[1]{V} zur Verf�gung. Des Weiteren wird mit einem Bedarf von \unit[2.000.000]{St�ck} gerechnet. Die Fertigungskosten sollten bezogen auf die St�ckzahl so gering wie m�glich ausfallen.

Diese Anforderungen sind nach Fest-, Mindest- und Wunschanforderung in Tabelle \ref{tab:Anforderungen} im Anhang aufgelistet.

\begin{table}[h]
	\centering
	\caption{Anforderungsliste eines Drucksensors}\label{tab:Anforderungen}
	\begin{tabular}{rrlp{2.2cm}p{1.2cm}p{3cm}}\toprule
		\textbf{Gliederung} & \textbf{Nr.} & \textbf{Bezeichnung} & \textbf{Werte,\newline Daten} & \textbf{Anf.-\newline Art} & \textbf{Quelle,\newline Bemerkungen}\\ \midrule
		\vspace{3pt} Geometrie &  1 & Abmessungen & \unit[10x10]{mm} & F & \\ \midrule
		\vspace{3pt} Kinematik &  2 & Prozessart & quasistatisch & F &\\ \midrule
		\vspace{3pt} Signale &  3 & Messbereich & \unit[0-1]{bar} & F &\\	
		\vspace{3pt}  &  4 & Genauigkeit & \unit[$\pm$50]{mbar} & M &\\
		\vspace{3pt}  &  5 & Ausgangsspannung & \unit[0-1]{V} & F &\\ \midrule
		\vspace{3pt} Energie &  6 & Zent. Spannungsvers. & \unit[$\pm$12]{V} & F &\\
		\vspace{3pt}  &  7 & Referenzspannung & \unit[1]{V} & F &\\ \midrule
		\vspace{3pt} Kosten &  8 & j�hrl. Bedarf & \unit[2000000]{St.} & W &\\
		\vspace{3pt}  &  9 & Fertigungskosten & so gering wie m�glich & W &\\ \bottomrule
	\end{tabular}
\end{table}

