%
%		Header-Datei
%

\documentclass[
	pdftex,             % Ausgabe des Latex-Dokuments als PDF
	12pt,				        % Schriftgroesse 12pt
	a4paper,		      	% Layout fuer Din A4
	german,			      	% deutsche Sprache, global
	BCOR10mm,		        % DIV-Wert fuer die Erstellung des Satzspiegels
	DIV14,				      % Korrektur f�r die Bindung
	parskip=half,       % Absatzgr��e -> halber Zeilenabstand
	oneside,			      % Einseitiger Druck
	%twoside,		        % Zweiseitiger Druck
	openany,            % Kapitel k�nnen auf geraden und ungeraden Seiten beginnen
	numbers=noenddot,   % Kapitelnummern immer ohne Punkt
	bibliography=totoc,	% Literaturverzeichnis im Inhaltsverzeichnis angeben
	index=totoc			    % Index im Inhaltsverzeichnis angeben
]{scrreprt}


% PDF-Dokument formatieren
\usepackage[
	pdftitle={\TITLE},
	pdfauthor={\AUTHOR},
	pdfcreator={pdfLatex},
	pdfsubject={\TYPE\ am Institut f�r Mikrotechnik der Technischen Universit�t Braunschweig},
	pdfkeywords={\TYPE, Technische Universit�t Braunschweig, TU-BS, Institut f�r Mikrotechnik, IMT, \TITLE, \AUTHOR, \MATRIKELNR},
  colorlinks={true}, % Umrandung von Links nicht sichtbar
  linkcolor={black},
  citecolor={black},
  filecolor={black},
  urlcolor={black},
	pdflang={de},
	hyperindex=true
]{hyperref}

% Deutsche Rechtschreibung
\usepackage{german,ngerman}
\usepackage[latin1]{inputenc}
\usepackage[T1]{fontenc}

% Aktuelles Datum ermitteln
\usepackage[ngerman]{datenumber}

% Erweiterung f�r ein deutsches Literaturverzeichnis
\bibliographystyle{alphadin}

% Benutzerdefinierte Kopf- und Fu�zeile
\usepackage[automark]{scrpage2}
\pagestyle{scrheadings}
\setheadsepline{.4pt} % Linie unter dem Header

% Erweiterte Mathematikbibliotheken
\usepackage{amsmath}
\usepackage{amssymb}

% Ma�einhaeiten-Darstellung verbessern
\usepackage{units}

% Einbinden von externen PDF Dateien
\usepackage[final]{pdfpages}

% Zum einbinden von Grafiken  
\usepackage{graphicx}